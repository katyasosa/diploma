\section{Количественная оценка метода}\label{compare}

Твиттер предоставляет API\footnote{https://dev.twitter.com/} для извлечения и поиска сообщений, в
том числе по поисковому запросу. Ответом на запрос к Твиттеру является набор сущностей, каждая из
которых хранит информацию о сообщении: его id, текст, имя пользователя-автора,
время публикации, а также, если оно является ответом или ретвитом другого, то указывается id
``родительского'' твита. В таком виде не восстановить цепочки твитов: чтобы найти диалог
между пользователями, нужно обойти все существующие твиты и найти из них те, которые ссылаются на
определённое сообщение, -- так можно найти все ответы на него или его ретвиты. Скорее всего, если
найден некоторый твит про объект, то ответы на него будут про этот же объект, то есть вычислять
ответы необходимо.

Предлагается делать это следующим образом. Пусть есть твит $T$, его id $T_{id}$, имя пользователя,
который его опубликовал $T_{user}$, и время публикации $T_{time}$. Отличительная особенность ответов
в Твиттере, как говорилось ранее, -- это упоминания, то есть ответ на твит пользователя с именем username
будут начинаться со строки ``@username''. Тогда, чтобы найти все ответы на твит $T$ будем искать не
по множеству всех возможнных сообщений, а по всем сообщениям, опубликованным позднее $T_{time}$ по
поисковому запросу ``@$T_{user}$''. Среди них уже можно будет выделить сообщения, которые ссылаются
на твит с номером $T_{id}$ -- это и будут все ответы на $T$.

Так собираются данные по слову-запросу для разметки эмоциональной окраски актуальных сообщений по
этой теме. Этот метод использовался для расширения тестовой выборки, используемой для анализа
алгоритмов в ходе всей работы.

Для обучения классификатора использовались 1000000 размеченных сообщений. Классификаторы сравнивались
на 386 тестовых примерах, 204 из которых отрицательные, 182 -- положительные.

\begin{table}[h]
    \centering
    \begin{tabular}{|c|c|c|c|c|}
      \hline
      \textbf{Метка класса} & \textbf{Precision} & \textbf{Recall} & \textbf{F1-score} &
      \textbf{Количество} \\ \hline
      -1.0&0.82&0.75&0.78&204\\ \hline
      1.0&0.74&0.82&0.78&182\\ \hline \hline
      avg / total&0.79&0.78&0.78&386\\
      \hline
    \end{tabular}
    \caption{Классификация наивным байесовским классификатором}\label{tab:nb1}
\end{table}
\begin{table}[h!]
    \centering
    \begin{tabular}{|c|c|c|c|c|}
      \hline
      \textbf{Метка класса} & \textbf{Precision} & \textbf{Recall} & \textbf{F1-score} &
      \textbf{Количество} \\ \hline
      -1.0&0.88&0.74&0.80&204\\ \hline
      1.0&0.75&0.88&0.81&182\\ \hline \hline
      avg / total&0.82&0.81&0.81&386\\
      \hline
    \end{tabular}
    \caption{Классификация байесовским наивным байесовским классификатором с триграммами и онтологиями}\label{tab:bnb}
\end{table}

В таблицах \ref{tab:nb1} и \ref{tab:bnb} приведено сравнение базового классификатора и его
изменённой версии, основанной на байесовском подходе и использовании триграмм и онтологий. Новый
классификатор дал улучшение предсказания на $3\%$, при этом метод не перестал быть инкрементально
обучающимся, то есть его можно уточнять в онлайн-режиме.
