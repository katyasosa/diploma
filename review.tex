\chapter{Обзор существующих решений}

\section{Конкретизация задачи}
Задача анализа эмоциональной окраски текстов сводится к задаче классификации. В
нашем случае имеется набор твитов, каждый из которых нужно отнести к одной из
трёх категорий: положительные, нейтральные или отрицательные.

Иногда классификация происходит в два этапа и на обоих этапах является бинарной.
На первом отделяются субъектиные сообщения от объективных. Объективными в этом
случае называются как раз те, которые не несут эмоциональной окраски и
явлются нейтральными варианте с тремя классами. Второй этап делит субъективные
тектсы на положительные и отрицательные. В случае с Твиттером, где почти все
сообщения субъективны, а критерии нейтральности можно сформулировать только в
смысле ``не положительное'' и ``не отрицательное'', будем использовать разделение
на три класса.

Твит -- это строка, состоящая из не более чем 140 символов. Он может содержать
специальные слова, начинающихся с определённых знаков: сразу после ``@'' пишется
имя пользователя, с которым сообщение связано или к которому оно обращено,
а после ``\#'' находится так называемый хештег -- слово, которое явно указывает
на связь твита с объектом, который этим словом обозначается. Все твиты создаются
пользователями, поэтому могут содержать опечатки, ошибки, сокращения,
особую пунктуацию и прочие способы выражения мысли в коротком тексте.
У каждого сообщения в Твиттере есть время, когда оно опубликовано, и автор.
Если один твит является ответом на другой, то у первого есть ещё и ссылка на второй,
то есть на ``родительский''. Ретвиты содержат также данные о первоначальном
размещении.

\section{Общий подход}
Вычислительно поставленная задача решается при помощи машинного обучения. Первое
упоминание анализа мнений в таком контексте относится к 2002 году. Тогда были рассмотрены
стандартные решения методом обучения с учителем \cite{pang2002thumbs} и
задачу для отзывов людей на специализированных ресурсах.
В первом случае за основу берутся лингвистические данные: словарь положительных
и словарь отрицательных слов. Во втором -- набор отзывов, уже разделённых на
положительные и отрицательные.