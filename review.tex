\chapter{Обзор существующих решений}

\section{Конкретизация задачи}
Задача анализа эмоциональной окраски текстов сводится к задаче классификации. В
нашем случае имеется набор твитов, каждый из которых нужно отнести к одной из
трёх категорий: положительные, нейтральные или отрицательные.

Иногда классификация происходит в два этапа и на обоих этапах является бинарной.
На первом отделяются субъектиные сообщения от объективных. Объективными в этом
случае называются как раз те, которые не несут эмоциональной окраски и
явлются нейтральными варианте с тремя классами. Второй этап делит субъективные
тектсы на положительные и отрицательные. В случае с Твиттером, где почти все
сообщения субъективны, а критерии нейтральности можно сформулировать только в
смысле ``не положительное'' и ``не отрицательное'', будем использовать разделение
на три класса.

Твит -- это строка, состоящая из не более чем 140 символов. Она может содержать
специальные слова, начинающиеся с определённых знаков: сразу после ``@'' пишется
имя пользователя, с которым сообщение связано или к которому оно обращено,
а после ``\#'' находится так называемый хештег -- слово, которое явно указывает
на связь твита с объектом, который этим словом обозначается. Все твиты создаются
пользователями, поэтому могут содержать опечатки, ошибки, сокращения,
особую пунктуацию и прочие способы выражения мысли в коротком тексте.
У каждого сообщения в Твиттере есть время, когда оно опубликовано, и автор.
Если один твит является ответом на другой, то у первого есть ещё и ссылка на второй,
то есть на ``родительский''. Ретвиты содержат также данные о первоначальном
размещении.

Вычислительно поставленная задача решается при помощи техник машинного обучения. Первое
упоминание анализа мнений относится к 2002 году. Тогда были рассмотрены
стандартные решения методом обучения без учителя \cite{turney2002thumbs} и методом обучения с
учителем \cite{pang2002thumbs}. В обеих статьях исследовались отзывы на
специализированном ресурсе: хотелось выяснить, рекомендует или нет пользователь, оставивший отзыв,
то, о чём он написал.

\section{Методы обучения без учителя для анализа мнений}
В статье \cite{pang2002thumbs} автор предлагает алгоритм обучения без учителя для классификации
отзывов на две категории: ``рекомендует'' и ``не рекомендует''. Алгоритм состоит из трёх
этапов.
\begin{enumerate}
\item Поиск словосочетаний с прилагательными или наречиями. Для дальнейшей работы алгоритма нужны
  будут фразы, где одно из слов -- прилагательное или наречие, а другое указывает на контекст. Если
  говорить про английский язык, то обычно для поиска второго слова достаточно взять соседнее.
\item Определение семантической ориентации словосочетания: положительное или отрициательное. На этом
  этапе используется PMM-IR алгоритм для выявления семантических ассоциаций
  \cite{Church:1989:PWA:1075434.1075449}. При помощи этого алгоритма автор определяет схожесть словосочетания ($phrase$)
  с ``excelent'' и с ``poor'' и вычисляет его семантическую ориентацию ($\operatorname{SO}$) по формуле:
$$\operatorname{SO}(phrase) = \operatorname{PMI}(phrase, \textrm{``excelent''}) -
\operatorname{PMI}(phrase, \textrm{``poor''})$$
где функция $\operatorname{PMI}(x,y)$ как раз определяет, есть ли семантическая ассоциация между $x$ и $y$. Для уточнения этой
формулы автор вводит отношение $\operatorname{NEAR}$ и функцию $\operatorname{hits}(x
\operatorname{NEAR} y)$ на основе $\operatorname{PMI}(x,y)$, которая показывает, попадает ли
$x$ в класс близких по смыслу к $y$ и считает семантическую ориентацию  по новой формуле:
$$
\operatorname{SO}(phrase)
= \log_2 \frac
          {\operatorname{hits}(phrase \operatorname{NEAR} \textrm{``excelent''}) \operatorname{hits}(\textrm{``poor''})}
          {\operatorname{hits}(phrase \operatorname{NEAR} \textrm{``poor''}) \operatorname{hits}(\textrm{``excelent''})}
$$
\item Определение семантической ориентации отзыва. Здесь считается средняя семантическая ориентация
  по всем словосочетаниям, найденным в отзыве, и определяется метка: ``рекомендует'', если среднее
  получилось положительным, и ``не рекомендует'', если оно получилось отрицательным.
\end{enumerate}
В результате алгоритм показывает точность около 80\% на отзывах, состоящих из нескольких
предложений, то есть представляющих собой полноценный текст. Сложность этого подхода в том, что для
работы второго этапа необходим корпус, собранный лингвистами вручную, то есть появляется бузесловный
человеческий фактор. Если вернуться к задаче для Твиттера, то особенности данных: опечатки,
зачастую отсутствие контекста, пролонгирование гласных и прочее -- обязыают постоянно расширять
словари для определения семантической ориентации, а раз это делает человек, то либо это невозможно,
либо составление такой или подобной базы нужно автоматизировать.

\section{Методы обучения с учителем для задачи анализа мнений}

\subsection{Общая формулировка}
Методы обучения с учителем предсказывают, к какому классу относится объект, на основании уже
размеченного набора данных, который также называется тренировочным. Каждый метод такого вида должен
уметь делать две вещи: обучаться на тренировочных данных и делать предсказание для новых. Слово
``обучиться'' здесь означает ``построить функцию, которая для примеров из тренировочного набора
сделает разметку, максимально близкую к действительной'', другими словами, нужно построить модель
данных.

В статье \cite{pang2002thumbs} авторы рассматривают три таких подхода: метод опорных
векторов, наивный байесовский классификатор и метод мультиномиальной регрессии. На каждом из них
сперва остановимся подробнее, а затем вернёмся к анализу мнений и статье.

\subsection{Наивный байесовский классификатор}
