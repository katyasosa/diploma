%% -*- ispell-language: russian -*-

\section*{Введение}

Не так давно грань между потребителями и создателями информации в интернете
исчезла: на смену статическим страницам у всех пользователей появилась
возможность публиковать свою информацию. Сейчас мы наблюдаем огромное
количество видов создаваемых материалов. Это может быть запись
в блоге или на форуме, фотография или видеозапись на соответствующем
ресурсе, отзыв в интернет-магазине,``статус'' в социальной сети и многое другое.
Совершенная простота размещения текстов от разных людей в одном месте
в интернете стала поводом для появления всевозможных вебсайтов, собирающих мнения
пользователей, например, о книгах, фильмах, товарах, и вот некоторые из них:
epinions.com, rottentomatoes.com, amazon.es, market.ya.ru. Прежде, чем что-то приоберсти,
покупатель ищет отзывы о серии необходимых товаров в интернете, он читает
десятки мнений различных людей, а на основании этих мнений делает вывод о том,
какой же продукт ему действительно подходит, и только после этого что-то покупает.
Со временем текстов стало так много, что обработать все их за разумное время человеку просто
не по силам. Именно такая ситуация стала причиной возникновения
задачи анализа мнений: появилась необходимость в создании системы для
автоматического поиска, классификации и представления точек зрения.

Анализ мнений -- одно из направлений области обработки текстов на естественных
языках. Саму задачу можно определить как вычислительное выявление
субъективности в текстах и отношения авторов этих текстов к некоторым объектам.
Изначально в качестве исследуемых данных использовались большие записи,
состоящие из нескольких предложений, в которых явно прослеживались связь и
контекст. Позже, с развитием социальных сетей, с появлением в них комментариев,
``статусов'' и  коротких сообщений, пользовательский контент стал менее ёмким,
но при этом более субъективным и превратился в бесконечный поток поступающей
информации. Ярким примером этому является сервис микроблогов
Twitter (http://twitter.com) (Твиттер). С помощью этого сервиса пользователи распространяют
свои взгляды на актуальные новости, связанные с разными интересными
другим людям областями, такими как политика, экономика, бизнес и другие,
рассказывают о купленных товарах, а также публикуют личную информацию, например,
что они сейчас делают и в каком настроении находятся.

В этой работе речь пойдёт именно о сообщениях, характерных для Твиттера.
Отличительная особенность этой платформы в том, что у пользователя есть
только 140 символов, чтобы выразить свои мысли или отношение к чему-либо.
Каждое сообщение, называемое здесь ``твит'' и публикуемое пользователем в
Твиттере, могут увидеть его подписчики -- люди, которые связаны с ним в этой
социальной сети. Подписчики (или иначе ``читатели'') могут быть как
односторонними, так и взаимными. Если человек увидел зинтересовавший его твит
и разместил его на своей странице, то говорят, что он ретвитнул
запись другого пользователя. В этом случае информация распространяется не только
на подписчиков первоначального автора, но и на читателей того, кто сделал
ретвит. Есть и другой вид взаимодействия пользовательской
информации: упоминания. Если читатель захотел ответить на какой-либо твит,
он это делает, вставив в начало своего сообщения псевдоним автора в Твиттере при помощи
символа @ (@username), тем самым, упоминая его. В этом случае ответ увидят
только те, кто читает обоих дискутирующих пользователей. Если упоминание
происходит в середине твита, то он доступен точно так же, как и обычный,
только упомянутому пользователю приходит отдельное оповещение.
Механизмов социального взаимодействия в Твиттере больше нет, но этого достаточно,
чтобы информация распространялась очень быстро и охватывала большую аудиторию.

Основной способ представления твитов -- это представление в виде ленты.
Пользователь, войдя на сайт, видит сообщения от всех читаемых им людей,
отсортированные в порядке удаления времени от настоящего момента. Дальше
он может перейти на страницу конкретного человека и прочитать только его
сообщения,  но обычно информация воспринимается именно в форме потока,
уходящего назад, до момента регистрации читающего пользователя на сайте.
Так он узнаёт, что нового произошло в жизни его знакомых, о чём рассказывают
интересные ему аккаунты и какие события обсуждаются в мире. При помощи ретвитов
информация действительно распространяется очень быстро. Например, можно
вспомнить ситуацию с (лучше какая-нибудь научная утка)

Кроме чтения релевантных сообщений от читаемых пользователей, можно
пользоваться поиском по хештегу. Хештег -- специальное слово, перед которым
стоит символ \# (\#хештег). Наличие хештега подразумевает, что твит имеет какое-то
отношение к объекту, обозначаему этим словом. Поиск по хештегам учитывает
записи всех пользователей, поэтому количество получаемой информации здесь не
просто большое, оно настолько велико, что страница с выдачей по популярным
запросам почти никогда не является актуальной. Твиты в выдаче также организованы
по принципу временной ленты, отдаляющейся от настоящего момента, поэтому, как
только пользователь пишет новый запрос, кто-то публикует запись с этим же хештегом.
Проблема даже не в том, что появляются новые твиты, а в том, что пользователь не
успевает обработать все существующие. Хештег -- это способ явно указать объект, о
котором идёт речь в сообщении, но не все пользователи их ставят, поэтому поиск
по ключевым словам даёт более полную информацию, хотя иногда и не носящую смысла,
а её количество уж тем более становится неподъёмным.

Что же можно делать с огромным количеством коротких текстов на определённую
тему, носящих, в основном, субъективный характер и не помещающихся вместе
в голове обычного человека? Точнее, что можно хотеть с ними делать? Вот пример:
выходит обновление какого-нибудь известного ПО, и компания публикует об этом
новость на своей странице в Твиттере. Читатели Твиттера этой компании воспринимают
сообщение о новой версии продукта и, во-первых, сами о ней узнают, во-вторых,
могут ретвитнуть её для своих подписчиков, и к аудитории новости присоединятся другие пользователи,
в-третьих, могут прокомментировать и показать тем самым своё отношение к событию.
На всех этапах распространения информации о продукте компании важно, какую
эмоциональную окраску она носит. В такой ситуации понятно, в какой момент и что
надо отслеживать. Но бывает, что человек написал в своём Твиттере мнение о
продукте независимо от публикаций компании или её представителей. Тут уже
начинается исследование эмоциональной окраски не среди комментариев к твитам и
ретвитам конкретной записи, а в целом среди текстов, имеющих отношение к целевому
объекту. Такая же задача встаёт, когда речь идёт об объектах из других областей:
всё те же политика, экономика, события в обществе и прочее.

Таким образом, ставится задача разметить в соответствии с эмоциональной окраской
множество твитов, имеющих отношение к конкретному объекту, заданному словом или
словосочетанием, с использованием особенностей именно этой социальной сети. Подобную
задачу решают и для больших текстов при помощи лигвистического словарного подхода и
вычислительно, методами машинного обучения. Цель данной работы --
исследовать проблему для Твиттера и предложить вариант её решения с использованием
аппарата вероятностных моделей. Для достижения этой цели можно сформулировать следующие шаги:
\begin{itemize}
\item проанализировать особенности задачи для микроблогов;
\item сравнить базовые методы обучения с учителем для данных из микроблогов и выбрать лучший по
  параметрам точности, полноты результатов и времени обучения;
\item предложить, обосновать и реализовать новый метод на основе выбранного;
\item оценить результаты работы нового метода.
\end{itemize}

Кроме Твиттера сервисами микроблогов отчасти явлюятся и другие социальные сети, например,
ВКонтакте (vk.com), Facebook (facebook.com), FourSquare (foursquare.com),
Instagram (instagram.com) поэтому задача, в целом, распространяема и на них,
но, так как эти платформы предоставляют много других возможностей, для ведения
микроблогов они используются гораздо меньше, чем Твиттер, и в этой работе
рассматриваться не будут.
