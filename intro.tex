\section*{Введение}

Не так давно грань между потребителями и создателями информации в интернете
исчезла: на смену статическим страницам у всех пользователей появилась
озможность публиковать свою информацию. Сейчас мы наблюдаем огромное
количество видов создаваемых пользователями материалов. Это может быть запись
в блоге или на форуме, фотография или видеозапись в соответствующем
ресурсе, отзыв в интернет-магазине,``статус'' в социальной сети и многое другое.
Совершенная простота размещения текстов от разных людей в одном месте
в интернете стала поводом для появления различных вебсайтов, собирающих мнения
пользователей, например, о книгах, фильмах, товарах, вот некоторые из них:
epinions.com, rottentomatoes.com, amazon.es. Прежде, чем что-то приоберсти,
покупатель ищет отзывы о серии необходимых товаров в интернете, он читает
десятки мнений различных людей, на основании этих мнений делает вывод о том,
какой же продукт ему действительно подходит, и только после этого что-то покупает.
Со временем текстов стало так много, что обработать все их человеку за разумное
время стало не по силам. Именно такая ситуация стала причиной возникновения
задачи анализа мнений: появилась необходимость в создании системы для
автоматического поиска, классификации и представления точек зрения.

Анализ мнений -- одна из задач области обработки текстов на естественных
языках. Саму задачу можно определить как вычислительное выявление
субъективности в текстах и отношения авторов этих текстов к некоторым объектам.
Изначально в качестве исследуемых данных использовались большие записи,
состоящие из нескольних предложений, в которых явно прослеживались связь и
контекст. Позже, с развитием социальных сетей, с появлением в них комментариев,
``статусов'' и  коротких сообщений, пользовательский контенет стал менее ёмким,
но при этом более субъективным и превратился в бесконечный поток поступающией
информации. Ярким примером этому является сервис микроблогов
Twitter (http://twitter.com) (Твиттер). С помощью этого сервиса пользователи распространяют
свои взгляды на актуальные новости, связанные с различными, интересными
другим людям областями, такими как политика, экономика, бизнес и другие,
рассказывают о купленных товарах, а также публикуют личную информацию, например,
то он сейчас чувствует или делает.

В этой работе речь пойдёт именно о сообщениях, характерных для Твиттера.
Отличительная особенность этой платформы в том, что у пользователя есть
только 140 символов, чтобы выразить свои мысли или отношение к чему-либо.
Каждое сообщение, называемое здесь ``твит'' и публикумое пользователем в
Твиттере, могут увидеть его подписчики -- люди, которые связаны с ним в этой
социальной сети. Подписчики (или иначе ``читатели'') могут быть как
односторонними, так и взаимными. Если человек увидел зинтересовавший его твит
и решил разместить его в своём профиле, то говорят, что он сделал ретвит
записи другого пользователя. В этом случае информация распространяется не только
на подписчиков первоначального автора, но и на читателей того, кто сделал
ретвит. Кроме ретвитов есть другой вид взаимодействия пользовательской
информации: упоминания. Если читатель захотел ответить на какой-либо твит,
он это делает, вставив в начало своего сообщения юзернэйм автора при помощи
символа @ (@username), тем самым, упоминая его. В этом случае ответ увидят
только те, кто читает обоих дискутирующих пользователей. Если упоминание
происходит в середине твита, то он доступен точно так же, как и обычный,
только упомянутому пользователю приходит отдельное оповещение.
Механизмов социального взаимодействия в Твиттере больше нет, но этого достаточно,
чтобы информация распространялась очень быстро и захватывала большую аудиторию.

Основной способ представления твитов -- представление в виде ленты.
Пользователь, войдя на сайт, видит сообщения от всех читаемых им людей,
отсортированные в порядке удаления времени от настоящего момента. Дальше
он может перейти на страницу конкретного человека и прочитать только его
сообщения,  но обычно информация воспринимается именно в форме потока,
уходящего назад, до момента регистрации читающего пользователя на сайте.
Так он узнаёт, что нового произошло в жизни его знакомых, о чём рассказывают
интересные ему аккаунты и какие события обсуждаются в мире. При помощи ретвитов
информация действительно распространяется очень быстро. Например, можно
вспомнить ситуация с (лучше какая-нибудь научная утка)

Кроме чтения релевантных сообщений от читаемых пользователей, можно
пользоваться поиском по хештегу. Хештег -- специальное слово, перед которым
стоит символ # (#хештег). Наличие хештега подразумевает, что твит имеет какое-то
отношение к объекту, обозначаему этим словом. Поиск по хештегам учитывает
записи всех пользователей, поэтому количество получаемой информации здесь не
просто большое, оно настолько велико, что страница с выдачей по популярным
запросам почти никогда не является актуальной. Твиты в выдаче также организованы
по принципу временной ленты, отдаляющейся от настоящего момента, поэтому, как
только пользователь пишет новый запрос, кто-то публикует запись с этим же хештегом.
Проблема даже не в том, что появляются новые твиты, а в том, что пользователь не
успевает обработать даже все существующие. Хештег -- это способ явно указать объект, о
котором идёт речь в сообщении, но не все пользователи их ставят, поэтому поиск
по ключевым словам даёт более полную информацию, хотя иногда и не носящую смысла.
