\section*{Заключение}
В ходе данной работы были решены поставленные задачи и достигнуты следующие результаты.
\begin{enumerate}
\item Проведено сравнение базовых методов обучения с учителем: наивный байесовский классификатор,
  метод опорных векторов и логистическая регрессия --- для данных из микроблогов по
  параметрам точности, полноты результатов и времени обучения. Подробнее о сравнении написано в
  разделе \ref{comparemeth}. Лучше всех себя показал наивный байесовский классификатор, время
  обучения которого на порядок ниже двух других.
\item Проанализированы особенности задачи анализа мнений для микроблогов, найдены варианты
  использования этих особенностей для улучшения методов классификации. Об использовании особенностей
  написано в разделе \ref{spec}.
\item На основе наивного байесовского классификатора предложен, обоснован и реализован новый
  метод. Для улучшения использовались: переход к байесовскому подходу (модель описана в разделе
  \ref{bnb}), триграммы для использования информации о связи слов в предложении (раздел \ref{ngram})
  и онотологии на базе категорий из Википедии (раздел \ref{ontology}).
\item Сравнение нового метода с базовым проведено на тех же данных, что и сравнение методов обучения
  с учителем. Разработанный метод дал улучшение на $3\%$ (раздел \ref{compare}).
\end{enumerate}

В качестве продолжения работы можно сформулировать и решить задачу классификации сообщений
из микроблогов на субъективные и объективные и такой классификацией дополнить уже
полученный метод. Кроме того, на основе предложенного в работе метода планируется
разработать веб-сайт, собирающий и демонстирующий пользователю статистику эмоциональной окраски сообщений из Твиттера
по поисковому запросу.
